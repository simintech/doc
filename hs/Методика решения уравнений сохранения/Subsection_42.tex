
\label{sec:subsection42}
Найдём конечно-разностный аналог уравнения сохранения импульса, записанный\linebreak для левой границы j-ой ячейки. Для этого рассмотрим уравнение сохранения импульса для гидравлической связи \eqref{formula225}. Заменим производную расхода по формуле дифференцирования назад
\begin{equation}
\label{formula45}
\frac{\partial G_j^{n+1}}{\partial\tau}=a_j^G + b_j^G \cdot dG_j^n.
\end{equation}

Запишем потери на трение и на преодоление местных сопротивлений в следующем виде:
\begin{equation}
\label{formula46}
\Delta P_{fr}+\Delta P_{loc}=\left(\frac{\lambda\cdot\frac L 2}{d_g} + \xi
\right)_{in}\cdot \frac{\rho_{in}\cdot w_{in}^2}{2} + \left(\frac{\lambda\cdot\frac L 2}{d_g} + \xi
\right)_{out}\cdot \frac{\rho_{out}\cdot w_{out}^2}{2}.
\end{equation} 

В этой формуле учитываются потери на трение на половине длины входной и выходной ячеек, а также потери на преодоление местных сопротивлений по обе стороны от гидравлической связи. Скорость жидкости можно переписать через массовый расход как $w=\frac{G}{\rho\cdot S}$. 

Запишем потери на трение в полунеявном виде с использованием массового расхода на текущем и на следующем шаге по времени в виде
\begin{eqnarray}
\label{formula47}
\Delta P_{fr}+\Delta P_{loc}=\left\{\left[\left(\frac{\lambda\cdot\frac L 2}{d_g} + \xi \right)\cdot \frac{1}{2\cdot\rho\cdot S^2} \right]_{in} + \left[\left(\frac{\lambda\cdot\frac L 2}{d_g} + \xi \right)\cdot \frac{1}{2\cdot\rho\cdot S^2} \right]_{out} \right\} \times \nonumber ~\\
\times | G_j^n | \cdot G_j^{n+1}=F_{fr}^n\cdot | G_j^n | \cdot G_j^{n+1},
\end{eqnarray}
где $F_{fr}^n$ --- приведённый коэффициент трения, рассчитанный по параметрам на текущем шаге по времени.

Запишем нивелирные потери (величину, обратную нивелирному напору) в следующем виде
\begin{equation}
\label{formula48}
\Delta P_{niv}=-H_{niv}=\rho_{in}^n\cdot g \cdot \frac{\Delta Z_{in}}{2}+\rho_{out}^n\cdot g \cdot \frac{\Delta Z_{out}}{2},
\end{equation}
где $\Delta Z_{in}$ и $\Delta Z_{out}$ --- изменение высоты в ячейке на входе гидравлической связи и на выходе гидравлической связи соответственно. Таким образом, если изменение высоты будет положительным, то необходимо будет затратить энергию на подъём жидкости при движении её в положительном направлении. 

Напор насоса запишем в виде
\begin{equation}
\label{formula49}
H_{pump}=\Delta P_{pump}+min\left(\frac{\partial P_{pump}}{\partial G_j},0\right)\cdot dG_j^n.
\end{equation}

Подставим~\eqref{formula45},~\eqref{formula47} и~\eqref{formula49} в~\eqref{formula225}. Получим
\begin{eqnarray}
\label{formula410}
J_j \cdot (a_j^G + b_j^G \cdot dG_j^n)  = 
P_{j-1}^{n+1} - P_j^{n+1} - F_{fr}^n\cdot | G_j^n | \cdot G_j^{n+1} - \Delta P_{acc} - \Delta P_{niv} + \nonumber ~\\
+ \Delta P_{pump}+min\left(\frac{\partial P_{pump}}{\partial G_j},0\right)\cdot dG_j^n. 	
\end{eqnarray}

Запишем вместо $G_j^{n+1}$ в правой части его выражение через приращение на шаге по времени и значение на предыдущем шаге: $G_j^{n+1}=G_j^n+dG_j^n$. Кроме того, поскольку все уравнения сохранения в теплогидравлическом коде решаются в отклонениях, запишем давления в ячейках, примыкающих к гидравлической связи, через их приращения и значения на предыдущем шаге по времени:
\begin{equation*}
\left\{
\begin{aligned}
	P_{j-1}^{n+1} & = P_{j-1}^n + dP_{j-1}^n; \\
	P_j^{n+1} & = P_j^n + dP_j^n.
\end{aligned}
\right.
\end{equation*}

Подставим эти выражения в~\eqref{formula410}:    
\begin{eqnarray}
\label{formula411}
J_j \cdot (a_j^G + b_j^G \cdot dG_j^n)  = 
(P_{j-1}^n + dP_{j-1}^n) - (P_j^n + dP_j^n) - F_{fr}^n\cdot | G_j^n | \cdot (G_j^n+dG_j^n) - \nonumber ~\\
- \Delta P_{acc} - \Delta P_{niv} + \Delta P_{pump}+min\left(\frac{\partial P_{pump}}{\partial G_j},0\right)\cdot dG_j^n. 	
\end{eqnarray}

Перегруппируем слагаемые и перепишем полученное уравнение в следующем \hphantom{aaa} виде
\begin{eqnarray}
\label{formula412}
-1\cdot dP_{j-1}^n+\left(J_j \cdot b_j^G + F_{fr}^n\cdot | G_j^n | - min\left(\frac{\partial P_{pump}}{\partial G_j},0\right)   \right)\cdot dG_j^n +1 \cdot dP_j^n + \nonumber ~\\
+ \left(J_j \cdot a_j^G - P_{j-1}^n + P_j^n + F_{fr}^n\cdot | G_j^n | \cdot G_j^n + \Delta P_{acc} + \Delta P_{niv} - \Delta P_{pump}  \right) = 0. 	
\end{eqnarray}

Получим окончательно конечно-разностное уравнение сохранения импульса для гидравлической связи в виде
\begin{equation}
\label{formula413}
\boxed{A_j^G \cdot dP_{j-1}^n + B_j^G \cdot dG_j^n + C_j^G \cdot dP_j^n + D_j^G = 0 = F_j^G},
\end{equation}
где $A_j^G=-1$; $B_j^G = J_j \cdot b_j^G + F_{fr}^n\cdot | G_j^n | - min\left(\frac{\partial P_{pump}}{\partial G_j},0\right) $; $C_j^G=1$;

\noindent $D_j^G = J_j \cdot a_j^G - P_{j-1}^n + P_j^n + F_{fr}^n\cdot | G_j^n | \cdot G_j^n + \Delta P_{acc} + \Delta P_{niv} - \Delta P_{pump} $. 
 





