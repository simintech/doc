
\label{sec:subsection41}

Найдём конечно-разностный аналог уравнения сохранения массы (для ячейки или для узла). Для этого рассмотрим уравнение сохранения массы для контрольного объёма \eqref{formula214}. При этом для упрощения записи отнесём объёмный источник массы $R_m$ к массовым расходам, входящим в контрольный объём. Получим
\begin{equation}
\label{formula41}
f_{comp}\cdot\left(\frac{\partial\rho}{\partial P}\right)_{h}\cdot \frac{\partial P}{\partial\tau} =
\frac 1 V \cdot \sum_{j=1}^{N} G_j + \rho\cdot W \cdot \frac 1 S \cdot \overline{\frac{\partial S}{\partial x}} - f_{comp}\cdot\left(\frac{\partial\rho}{\partial h}\right)_{P}\cdot \frac{\partial h}{\partial\tau}.
\end{equation}

Запишем производные энтальпии и давления по формуле дифференцирования назад:
$$
\left\{
\begin{aligned}
\frac{\partial P_j^{n+1}}{\partial\tau} & = a_j^P + b_j^P \cdot dP_j^n; \\
\frac{\partial h_j^{n+1}}{\partial\tau} & = a_j^h + b_j^h \cdot dh_j^n.
\end{aligned}
\right.
$$

Сумму расходов разделим на сумму входящих расходов и сумму выходящих:
$$
\sum_{j=1}^{N} G_j=\sum_{j=1}^{N_{in}} G_j - \sum_{j=1}^{N_{out}} G_j.
$$

Каждый из расходов запишем в виде суммы расхода на предыдущем шаге по времени и восходящую разность как $G_j^{n+1}=G_j^n+dG_j^n$. Подставим всё это в уравнение~\eqref{formula41}. Получим
\begin{eqnarray}
\label{formula42}
f_{comp}\cdot\left(\frac{\partial\rho}{\partial P}\right)_{h}\cdot (a_j^P + b_j^P \cdot dP_j^n) =
\frac 1 V \cdot \left(\sum_{j=1}^{N_{in}} (G_j^n+dG_j^n) - \sum_{j=1}^{N_{out}} (G_j^n+dG_j^n)\right) + \nonumber ~\\
+ \rho\cdot W \cdot \frac 1 S \cdot \overline{\frac{\partial S}{\partial x}} - f_{comp}\cdot\left(\frac{\partial\rho}{\partial h}\right)_{P}\cdot (a_j^h + b_j^h \cdot dh_j^n).
\end{eqnarray}

Умножим обе части уравнения на объём ячейки $V$ и перегруппируем слагаемые следующим образом
\begin{eqnarray}
\label{formula43}
-1 \cdot \sum_{j=1}^{N_{in}} dG_j^n + V \cdot f_{comp}\cdot \left( \frac{\partial\rho}{\partial P}\right)_{h} \cdot b_j^P \cdot dP_j^n + 1 \cdot \sum_{j=1}^{N_{out}} dG_j^n + V \cdot f_{comp}\cdot\left(\frac{\partial\rho}{\partial h}\right)_{P} \cdot b_j^h \cdot dh_j^n + \nonumber ~\\
+ \left(V \cdot f_{comp}\cdot \left( \frac{\partial\rho}{\partial P}\right)_{h} \cdot a_j^P + V \cdot f_{comp}\cdot \left(\frac{\partial\rho}{\partial h}\right)_{P}\cdot a_j^h - \rho\cdot W \cdot L \cdot \overline{\frac{\partial S}{\partial x}} - \sum_{j=1}^{N_{in}} G_j^n + \sum_{j=1}^{N_{out}} G_j^n \right) = 0. 
\end{eqnarray}

Получим окончательно конечно-разностное уравнение сохранения массы для \\ кон\-троль\-но\-го объёма в виде
\begin{equation}
\label{formula44}
\boxed{A_j^P \cdot \sum_{j=1}^{N_{in}} dG_j^n + B_j^P \cdot dP_j^n + C_j^P \cdot \sum_{j=1}^{N_{out}} dG_j^n + D_j^P \cdot dh_j^n + E_j^P = 0 = F_j^P},
\end{equation}
где $A_j^P=-1$; $B_j^P=V \cdot f_{comp}\cdot \left( \frac{\partial\rho}{\partial P}\right)_{h} \cdot b_j^P$; $C_j^P=1$; $D_j^P=V \cdot f_{comp}\cdot \left(\frac{\partial\rho}{\partial h}\right)_{P} \cdot b_j^h$;

\noindent $E_j^P=V \cdot f_{comp}\cdot \left( \frac{\partial\rho}{\partial P}\right)_{h} \cdot a_j^P + V \cdot f_{comp}\cdot \left(\frac{\partial\rho}{\partial h}\right)_{P}\cdot a_j^h - \rho\cdot W \cdot L \cdot \overline{\frac{\partial S}{\partial x}} - \sum_{j=1}^{N_{in}} G_j^n + \sum_{j=1}^{N_{out}} G_j^n $.










