
\section{Разностные уравнения сохранения}
\label{sec:section4}
При конечно-разностной аппроксимации уравнений сохранения реализованы следующие основные принципы:
\begin{itemize}[topsep=5pt, itemsep=-3pt]
\item[---] в основе пространственной дискретизации уравнений лежит метод контрольного объёма. При этом разностные уравнения сохранения массы и энергии составляются балансным методом в консервативной форме;
\item[---] по пространственной координате используется смещённая (шахматная) сетка. Схема разбиения показана на рисунке~\ref{fig11};
\item[---] аппроксимация конвективных членов переноса массы и энергии осуществлена по схеме против потока;
\item[---] используется полунеявная численная схема аппроксимации по времени;
\item[---] решение общей теплогидравлической задачи "расщеплено"\  на последовательное решение гидравлической и тепловой задач.  	
\end{itemize}
\subsection{Уравнение сохранения массы}	

\label{sec:subsection41}

Найдём конечно-разностный аналог уравнения сохранения массы (для ячейки или для узла). Для этого рассмотрим уравнение сохранения массы для контрольного объёма \eqref{formula214}. При этом для упрощения записи отнесём объёмный источник массы $R_m$ к массовым расходам, входящим в контрольный объём. Получим
\begin{equation}
\label{formula41}
f_{comp}\cdot\left(\frac{\partial\rho}{\partial P}\right)_{h}\cdot \frac{\partial P}{\partial\tau} =
\frac 1 V \cdot \sum_{j=1}^{N} G_j + \rho\cdot W \cdot \frac 1 S \cdot \overline{\frac{\partial S}{\partial x}} - f_{comp}\cdot\left(\frac{\partial\rho}{\partial h}\right)_{P}\cdot \frac{\partial h}{\partial\tau}.
\end{equation}

Запишем производные энтальпии и давления по формуле дифференцирования назад:
$$
\left\{
\begin{aligned}
\frac{\partial P_j^{n+1}}{\partial\tau} & = a_j^P + b_j^P \cdot dP_j^n; \\
\frac{\partial h_j^{n+1}}{\partial\tau} & = a_j^h + b_j^h \cdot dh_j^n.
\end{aligned}
\right.
$$

Сумму расходов разделим на сумму входящих расходов и сумму выходящих:
$$
\sum_{j=1}^{N} G_j=\sum_{j=1}^{N_{in}} G_j - \sum_{j=1}^{N_{out}} G_j.
$$

Каждый из расходов запишем в виде суммы расхода на предыдущем шаге по времени и восходящую разность как $G_j^{n+1}=G_j^n+dG_j^n$. Подставим всё это в уравнение~\eqref{formula41}. Получим
\begin{eqnarray}
\label{formula42}
f_{comp}\cdot\left(\frac{\partial\rho}{\partial P}\right)_{h}\cdot (a_j^P + b_j^P \cdot dP_j^n) =
\frac 1 V \cdot \left(\sum_{j=1}^{N_{in}} (G_j^n+dG_j^n) - \sum_{j=1}^{N_{out}} (G_j^n+dG_j^n)\right) + \nonumber ~\\
+ \rho\cdot W \cdot \frac 1 S \cdot \overline{\frac{\partial S}{\partial x}} - f_{comp}\cdot\left(\frac{\partial\rho}{\partial h}\right)_{P}\cdot (a_j^h + b_j^h \cdot dh_j^n).
\end{eqnarray}

Умножим обе части уравнения на объём ячейки $V$ и перегруппируем слагаемые следующим образом
\begin{eqnarray}
\label{formula43}
-1 \cdot \sum_{j=1}^{N_{in}} dG_j^n + V \cdot f_{comp}\cdot \left( \frac{\partial\rho}{\partial P}\right)_{h} \cdot b_j^P \cdot dP_j^n + 1 \cdot \sum_{j=1}^{N_{out}} dG_j^n + V \cdot f_{comp}\cdot\left(\frac{\partial\rho}{\partial h}\right)_{P} \cdot b_j^h \cdot dh_j^n + \nonumber ~\\
+ \left(V \cdot f_{comp}\cdot \left( \frac{\partial\rho}{\partial P}\right)_{h} \cdot a_j^P + V \cdot f_{comp}\cdot \left(\frac{\partial\rho}{\partial h}\right)_{P}\cdot a_j^h - \rho\cdot W \cdot L \cdot \overline{\frac{\partial S}{\partial x}} - \sum_{j=1}^{N_{in}} G_j^n + \sum_{j=1}^{N_{out}} G_j^n \right) = 0. 
\end{eqnarray}

Получим окончательно конечно-разностное уравнение сохранения массы для \\ кон\-троль\-но\-го объёма в виде
\begin{equation}
\label{formula44}
\boxed{A_j^P \cdot \sum_{j=1}^{N_{in}} dG_j^n + B_j^P \cdot dP_j^n + C_j^P \cdot \sum_{j=1}^{N_{out}} dG_j^n + D_j^P \cdot dh_j^n + E_j^P = 0 = F_j^P},
\end{equation}
где $A_j^P=-1$; $B_j^P=V \cdot f_{comp}\cdot \left( \frac{\partial\rho}{\partial P}\right)_{h} \cdot b_j^P$; $C_j^P=1$; $D_j^P=V \cdot f_{comp}\cdot \left(\frac{\partial\rho}{\partial h}\right)_{P} \cdot b_j^h$;

\noindent $E_j^P=V \cdot f_{comp}\cdot \left( \frac{\partial\rho}{\partial P}\right)_{h} \cdot a_j^P + V \cdot f_{comp}\cdot \left(\frac{\partial\rho}{\partial h}\right)_{P}\cdot a_j^h - \rho\cdot W \cdot L \cdot \overline{\frac{\partial S}{\partial x}} - \sum_{j=1}^{N_{in}} G_j^n + \sum_{j=1}^{N_{out}} G_j^n $.












\subsection{Уравнение сохранения импульса}

\label{sec:subsection42}
Найдём конечно-разностный аналог уравнения сохранения импульса, записанный\linebreak для левой границы j-ой ячейки. Для этого рассмотрим уравнение сохранения импульса для гидравлической связи \eqref{formula225}. Заменим производную расхода по формуле дифференцирования назад
\begin{equation}
\label{formula45}
\frac{\partial G_j^{n+1}}{\partial\tau}=a_j^G + b_j^G \cdot dG_j^n.
\end{equation}

Запишем потери на трение и на преодоление местных сопротивлений в следующем виде:
\begin{equation}
\label{formula46}
\Delta P_{fr}+\Delta P_{loc}=\left(\frac{\lambda\cdot\frac L 2}{d_g} + \xi
\right)_{in}\cdot \frac{\rho_{in}\cdot w_{in}^2}{2} + \left(\frac{\lambda\cdot\frac L 2}{d_g} + \xi
\right)_{out}\cdot \frac{\rho_{out}\cdot w_{out}^2}{2}.
\end{equation} 

В этой формуле учитываются потери на трение на половине длины входной и выходной ячеек, а также потери на преодоление местных сопротивлений по обе стороны от гидравлической связи. Скорость жидкости можно переписать через массовый расход как $w=\frac{G}{\rho\cdot S}$. 

Запишем потери на трение в полунеявном виде с использованием массового расхода на текущем и на следующем шаге по времени в виде
\begin{eqnarray}
\label{formula47}
\Delta P_{fr}+\Delta P_{loc}=\left\{\left[\left(\frac{\lambda\cdot\frac L 2}{d_g} + \xi \right)\cdot \frac{1}{2\cdot\rho\cdot S^2} \right]_{in} + \left[\left(\frac{\lambda\cdot\frac L 2}{d_g} + \xi \right)\cdot \frac{1}{2\cdot\rho\cdot S^2} \right]_{out} \right\} \times \nonumber ~\\
\times | G_j^n | \cdot G_j^{n+1}=F_{fr}^n\cdot | G_j^n | \cdot G_j^{n+1},
\end{eqnarray}
где $F_{fr}^n$ --- приведённый коэффициент трения, рассчитанный по параметрам на текущем шаге по времени.

Запишем нивелирные потери (величину, обратную нивелирному напору) в следующем виде
\begin{equation}
\label{formula48}
\Delta P_{niv}=-H_{niv}=\rho_{in}^n\cdot g \cdot \frac{\Delta Z_{in}}{2}+\rho_{out}^n\cdot g \cdot \frac{\Delta Z_{out}}{2},
\end{equation}
где $\Delta Z_{in}$ и $\Delta Z_{out}$ --- изменение высоты в ячейке на входе гидравлической связи и на выходе гидравлической связи соответственно. Таким образом, если изменение высоты будет положительным, то необходимо будет затратить энергию на подъём жидкости при движении её в положительном направлении. 

Напор насоса запишем в виде
\begin{equation}
\label{formula49}
H_{pump}=\Delta P_{pump}+min\left(\frac{\partial P_{pump}}{\partial G_j},0\right)\cdot dG_j^n.
\end{equation}

Подставим~\eqref{formula45},~\eqref{formula47} и~\eqref{formula49} в~\eqref{formula225}. Получим
\begin{eqnarray}
\label{formula410}
J_j \cdot (a_j^G + b_j^G \cdot dG_j^n)  = 
P_{j-1}^{n+1} - P_j^{n+1} - F_{fr}^n\cdot | G_j^n | \cdot G_j^{n+1} - \Delta P_{acc} - \Delta P_{niv} + \nonumber ~\\
+ \Delta P_{pump}+min\left(\frac{\partial P_{pump}}{\partial G_j},0\right)\cdot dG_j^n. 	
\end{eqnarray}

Запишем вместо $G_j^{n+1}$ в правой части его выражение через приращение на шаге по времени и значение на предыдущем шаге: $G_j^{n+1}=G_j^n+dG_j^n$. Кроме того, поскольку все уравнения сохранения в теплогидравлическом коде решаются в отклонениях, запишем давления в ячейках, примыкающих к гидравлической связи, через их приращения и значения на предыдущем шаге по времени:
\begin{equation*}
\left\{
\begin{aligned}
	P_{j-1}^{n+1} & = P_{j-1}^n + dP_{j-1}^n; \\
	P_j^{n+1} & = P_j^n + dP_j^n.
\end{aligned}
\right.
\end{equation*}

Подставим эти выражения в~\eqref{formula410}:    
\begin{eqnarray}
\label{formula411}
J_j \cdot (a_j^G + b_j^G \cdot dG_j^n)  = 
(P_{j-1}^n + dP_{j-1}^n) - (P_j^n + dP_j^n) - F_{fr}^n\cdot | G_j^n | \cdot (G_j^n+dG_j^n) - \nonumber ~\\
- \Delta P_{acc} - \Delta P_{niv} + \Delta P_{pump}+min\left(\frac{\partial P_{pump}}{\partial G_j},0\right)\cdot dG_j^n. 	
\end{eqnarray}

Перегруппируем слагаемые и перепишем полученное уравнение в следующем \hphantom{aaa} виде
\begin{eqnarray}
\label{formula412}
-1\cdot dP_{j-1}^n+\left(J_j \cdot b_j^G + F_{fr}^n\cdot | G_j^n | - min\left(\frac{\partial P_{pump}}{\partial G_j},0\right)   \right)\cdot dG_j^n +1 \cdot dP_j^n + \nonumber ~\\
+ \left(J_j \cdot a_j^G - P_{j-1}^n + P_j^n + F_{fr}^n\cdot | G_j^n | \cdot G_j^n + \Delta P_{acc} + \Delta P_{niv} - \Delta P_{pump}  \right) = 0. 	
\end{eqnarray}

Получим окончательно конечно-разностное уравнение сохранения импульса для гидравлической связи в виде
\begin{equation}
\label{formula413}
\boxed{A_j^G \cdot dP_{j-1}^n + B_j^G \cdot dG_j^n + C_j^G \cdot dP_j^n + D_j^G = 0 = F_j^G},
\end{equation}
где $A_j^G=-1$; $B_j^G = J_j \cdot b_j^G + F_{fr}^n\cdot | G_j^n | - min\left(\frac{\partial P_{pump}}{\partial G_j},0\right) $; $C_j^G=1$;

\noindent $D_j^G = J_j \cdot a_j^G - P_{j-1}^n + P_j^n + F_{fr}^n\cdot | G_j^n | \cdot G_j^n + \Delta P_{acc} + \Delta P_{niv} - \Delta P_{pump} $. 
 







\subsection{Уравнение сохранения энергии}

Найдём конечно-разностный аналог уравнения сохранения энергии. Для этого рассмотрим уравнение для контрольного объёма (ячейки либо узла) в виде~\eqref{formula237}. Заменим производные энтальпии и давления по формуле дифференцирования назад. Перепишем сумму, содержащую сумму расходов в гидравлических связях, следующим образом:
\begin{equation}
\label{formula414}
\sum_{k=1}^{N_{gc}} \mu_k \cdot G_k \cdot (h_k-h_j) = \sum_{k=1}^{N_{in}} \mu_k \cdot G_k \cdot (h_k-h_j) - \sum_{k=1}^{N_{out}} (1-\mu_k) \cdot G_k \cdot (h_k-h_j).
\end{equation}

Таким образом, если выходящий расход больше нуля, то во второй сумме $\mu_k=1$ и соответствующее слагаемое обнуляется, а если выходящий расход меньше нуля, то он становится входящим и вносит свой вклад в общую сумму.

Расходы запишем через сумму расходов на предыдущем шаге по времени и соответствующей восходящей разности. Кроме того, отнесём источник массы $R_m$ в соответствующую сумму расходов (в зависимости от его знака). Сумму всех входящих в контрольный объём тепловых потоков для упрощения запишем в виде $$V\cdot Q + Q_{wall} + Q_{ax} = Q+min\left(\frac{\partial Q}{\partial h_j^n},0 \right)\cdot dh_j^n.$$ В итоге получим: 
\begin{align}
\label{formula415}
&\rho\cdot V\cdot (a_j^h + b_j^h \cdot dh_j^n)=\sum_{k=1}^{N_{in}} \mu_k \cdot (G_k^n+dG_k^n) \cdot (h_k-h_j) - \nonumber ~\\
- &\sum_{k=1}^{N_{out}} (1-\mu_k) \cdot (G_k^n+dG_k^n) \cdot (h_k-h_j) + V\cdot (a_j^P+b_j^P\cdot dP_j^n)
+Q+min\left(\frac{\partial Q}{\partial h_j^n},0 \right)\cdot dh_j^n.
\end{align}

Перегруппируем слагаемые следующим образом:
\begin{eqnarray}
\label{formula416}
-\sum_{k=1}^{N_{in}} \mu_k \cdot (h_k-h_j) \cdot dG_k^n + \left(\rho\cdot V\cdot b_j^h - min\left(\frac{\partial Q}{\partial h_j^n},0 \right) \right) \cdot dh_j^n + \nonumber ~\\
+ \sum_{k=1}^{N_{out}} (1-\mu_k) \cdot (h_k-h_j) \cdot dG_k^n - V \cdot b_j^P \cdot dP_j^n + \bigg(\rho\cdot V\cdot a_j^h - \nonumber ~\\
- \sum_{k=1}^{N_{in}} \mu_k \cdot G_k^n \cdot (h_k-h_j) + \sum_{k=1}^{N_{out}} (1-\mu_k) \cdot G_k^n \cdot (h_k-h_j) - V \cdot a_j^P -Q \bigg)=0.  
\end{eqnarray}

Получим окончательно конечно-разностное уравнение сохранения энергии для контрольного объёма в виде
\begin{equation}
\label{formula417}
\boxed{\sum_{k=1}^{N_{in}} A_{kj}^h \cdot dG_k^n + B_j^h \cdot dh_j^n + \sum_{k=1}^{N_{out}} C_{kj}^h \cdot dG_k^n + D_j^h \cdot dP_j^n + E_j^h = 0 = F_j^h},
\end{equation}
где $A_{kj}^h=-\mu_k\cdot (h_k-h_j)$; $B_j^h=\rho\cdot V\cdot b_j^h - min\left(\frac{\partial Q}{\partial h_j^n},0 \right)$; $C_{kj}^h=(1-\mu_k)\cdot (h_k-h_j)$;
$D_j^h=-V \cdot b_j^P$; 

\noindent $E_j^h=\rho\cdot V\cdot a_j^h -\sum_{k=1}^{N_{in}} \mu_k \cdot G_k^n \cdot (h_k-h_j) + \sum_{k=1}^{N_{out}} (1-\mu_k) \cdot G_k^n \cdot (h_k-h_j) - V \cdot a_j^P -Q$.

В полученные дискретных аналогах~\eqref{formula44}, \eqref{formula413} и \eqref{formula417} член $F_j$ равен разности левой и правой части соответствующего уравнения сохранения.  












\newpage
